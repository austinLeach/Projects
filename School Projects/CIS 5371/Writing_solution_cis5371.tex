%
% This is the LaTeX template file for lecture notes for CS294-8,
% Computational Biology for Computer Scientists.  When preparing 
% LaTeX notes for this class, please use this template.
%
% To familiarize yourself with this template, the body contains
% some examples of its use.  Look them over.  Then you can
% run LaTeX on this file.  After you have LaTeXed this file then
% you can look over the result either by printing it out with
% dvips or using xdvi.
%
% This template is based on the template for Prof. Sinclair's CS 270.

\documentclass[twoside]{article}
\usepackage{graphicx,xspace,amsfonts}
\usepackage{setspace}
\setlength{\oddsidemargin}{0.25 in}
\setlength{\evensidemargin}{-0.25 in}
\setlength{\topmargin}{-0.6 in}
\setlength{\textwidth}{6.5 in}
\setlength{\textheight}{8.5 in}
\setlength{\headsep}{0.75 in}
\setlength{\parindent}{0 in}
\setlength{\parskip}{0.1 in}

%
% The following commands set up the lecnum (lecture number)
% counter and make various numbering schemes work relative
% to the lecture number.
%
\newcounter{lecnum}
\renewcommand{\thepage}{\thelecnum-\arabic{page}}
\renewcommand{\thesection}{\thelecnum.\arabic{section}}
\renewcommand{\theequation}{\thelecnum.\arabic{equation}}
\renewcommand{\thefigure}{\thelecnum.\arabic{figure}}
\renewcommand{\thetable}{\thelecnum.\arabic{table}}

\newcommand{\figref}[1]{Figure~\ref{#1}}


\newcommand{\ALGORITHM}{\mbox{\kwfont algorithm}\xspace}
\newcommand{\AND}{\mbox{\kwfont and}\xspace}
\newcommand{\NOT}{\mbox{\kwfont not}\xspace}
\newcommand{\AS}{\mbox{\kwfont as}\xspace}
\newcommand{\DO}{\mbox{\kwfont do}\xspace}
\newcommand{\TO}{\mbox{\kwfont to}\xspace}
\newcommand{\IF}{\mbox{\kwfont if}\xspace}
\newcommand{\FI}{\mbox{\kwfont fi}\xspace}
\newcommand{\ELSE}{\mbox{\kwfont else}\xspace}
\newcommand{\ELSEIF}{\mbox{\kwfont elsif}\xspace}
\newcommand{\FOR}{\mbox{\kwfont for}\xspace}
\newcommand{\OD}{\mbox{\kwfont od}\xspace}
\newcommand{\OR}{\mbox{\kwfont or}\xspace}
\newcommand{\PARSE}{\mbox{\kwfont parse}\xspace}
\newcommand{\PROCEDURE}{\mbox{\kwfont procedure}\xspace}
\newcommand{\PRIVATE}{\mbox{\kwfont private}\xspace}
\newcommand{\RETURN}{\mbox{\kwfont return}\xspace}
\newcommand{\SELECT}{\mbox{\kwfont select}\xspace}
\newcommand{\THEN}{\mbox{\kwfont then}\xspace}
\newcommand{\WHILE}{\mbox{\kwfont while}\xspace}

\newcommand{\kwfont}{\bf}


\DeclareMathAlphabet{\mathsl}{OT1}{cmr}{m}{sl}
\DeclareMathAlphabet{\mathsc}{OT1}{cmr}{m}{sc}
\DeclareMathAlphabet{\mathslbf}{OT1}{cmr}{bx}{sl}
% math script font; extra commands to make slightly larger
\DeclareFontFamily{OT1}{pzc}{}
\DeclareFontShape{OT1}{pzc}{m}{it}%
             {<-> s * [1.150] pzcmi7t}{}
\DeclareMathAlphabet{\mathscript}{OT1}{pzc}{m}{it}

%
% The following macro is used to generate the header.
%
\newcommand{\lecture}[4]{
   \pagestyle{myheadings}
   \thispagestyle{plain}
   \newpage
   \setcounter{lecnum}{#1}
   \setcounter{page}{1}
   \noindent
   \begin{center}
   \framebox{
      \vbox{\vspace{2mm}
    \hbox to 6.28in { {\bf CIS 5371 -- Cryptography
                        \hfill } }
       \vspace{4mm}
       \hbox to 6.28in { {\Large \hfill Writing Exercise   \hfill} }
       \vspace{2mm}
       \hbox to 6.28in { {\it Instructor: #3 \hfill } }
      \vspace{2mm}}
   }
   \end{center}
   \markboth{Homework #1: #2}{Lecture #1: #2}
      \vspace*{4mm}
}

%
% Convention for citations is authors' initials followed by the year.
% For example, to cite a paper by Leighton and Maggs you would type
% \cite{LM89}, and to cite a paper by Strassen you would type \cite{S69}.
% (To avoid bibliography problems, for now we redefine the \cite command.)
% Also commands that create a suitable format for the reference list.
\renewcommand{\cite}[1]{[#1]}
\def\beginrefs{\begin{list}%
        {[\arabic{equation}]}{\usecounter{equation}
         \setlength{\leftmargin}{2.0truecm}\setlength{\labelsep}{0.4truecm}%
         \setlength{\labelwidth}{1.6truecm}}}
\def\endrefs{\end{list}}
\def\bibentry#1{\item[\hbox{[#1]}]}

%Use this command for a figure; it puts a figure in wherever you want it.
%usage: \fig{NUMBER}{SPACE-IN-INCHES}{CAPTION}
\newcommand{\fig}[3]{
			\vspace{#2}
			\begin{center}
			Figure \thelecnum.#1:~#3
			\end{center}
	}
% Use these for theorems, lemmas, proofs, etc.
\newtheorem{theorem}{Theorem}[lecnum]
\newtheorem{lemma}[theorem]{Lemma}
\newtheorem{proposition}[theorem]{Proposition}
\newtheorem{claim}[theorem]{Claim}
\newtheorem{corollary}[theorem]{Corollary}
\newtheorem{definition}[theorem]{Definition}
\newtheorem{problem}[theorem]{Exercise}

\newenvironment{proof}{{\bf Proof:}}{\hfill\rule{2mm}{2mm}}

\newcommand{\headingquestion} [1]{\vspace{10pt}\noindent\textsc{#1?}}
\newcommand{\noskipheading}[1]{\noindent\textsc{#1.}}
\newcommand{\noskipheadingquestion}[1]{\noindent\textsc{#1?}}
\newcommand{\heading}[1]{\vspace{10pt}\noindent\textsc{#1.}}
\newcommand{\Heading}[1]{\vspace{10pt}\noindent\textit{#1.}}
\newcommand{\headingg}[1]{\vspace{10pt}\noindent\textsc{#1}}
\newcommand{\Headingg}[1]{\vspace{10pt}\noindent\textbf{#1.}}

\newcommand{\calL}{{\mathcal L}}
\newcommand{\vecB}{\mathbf{B}}
\newcommand{\vecu}{\mathbf{u}}
\newcommand{\vecv}{\mathbf{v}}
\newcommand{\vecz}{\mathbf{z}}
\newcommand{\veca}{\mathbf{a}}
\newcommand{\vecw}{\mathbf{w}}


\newcommand{\vectorr}[1]{\stackrel{\longrightarrow}{{#1}}}
\newcommand{\xor}{\oplus}

\newcommand{\calK}{{\mathcal K}}
\newcommand{\calG}{{\mathcal G}}
\newcommand{\calE}{{\mathcal E}}
\newcommand{\calD}{{\mathcal D}}
\newcommand{\calM}{{\mathcal M}}
\newcommand{\getsr}{{\:{\leftarrow{\hspace*{-3pt}\raisebox{.75pt}{$\scriptscriptstyle\$$}}}\:}}
\newcommand{\bits}{\{0, 1\}}


\newcommand{\Adv}{\mathbf{Adv}}
\newcommand{\prg}{\mathrm{prg}}
\newcommand{\concat}{\|}


\newcommand{\twoCols}[4]{
\begin{center}
        \framebox{
        \begin{tabular}{c@{\hspace*{.4em}}|c@{\hspace*{.4em}}c}
        \begin{minipage}[t]{#1\textwidth}\setstretch{1.2}\gamesfontsize #3 \end{minipage}
        &
        \begin{minipage}[t]{#2\textwidth}\setstretch{1.2}\gamesfontsize #4 \end{minipage}
        \end{tabular}
        }
\end{center}
}


\newcommand{\gamesfontsize}{\small}


\newcommand{\threeCols}[6]{
\begin{center}
        \framebox{
        \begin{tabular}{@{\hspace{-0.2em}}c@{\hspace{0.2em}}|@{\hspace{0.2em}}c@{\hspace{0.2em}}|@{\hspace{0.2em}}c@{\hspace{0.2em}}}
        \begin{minipage}[t]{#1\textwidth}\setstretch{1.1}\gamesfontsize #4
        \end{minipage} &
        \begin{minipage}[t]{#2\textwidth}\setstretch{1.1}\gamesfontsize #5
        \end{minipage} &
        \begin{minipage}[t]{#3\textwidth}\setstretch{1.1}\gamesfontsize #6
        \end{minipage}
        \end{tabular}
        }
\end{center}
}


\newcommand{\gReal}{\mathsc{Real}}
\newcommand{\gIdeal}{\mathsc{Ideal}}



% **** IF YOU WANT TO DEFINE ADDITIONAL MACROS FOR YOURSELF, PUT THEM HERE:

\begin{document}
%FILL IN THE RIGHT INFO.
%\lecture{**LECTURE-NUMBER**}{**DATE**}{**LECTURER**}{**SCRIBE**}
\lecture{3}{}{Viet Tung Hoang}{}
%\footnotetext{These notes are partially based on those of Nigel Mansell.}

% **** YOUR NOTES GO HERE:

% Some general latex examples and examples making use of the
% macros follow.  
%**** IN GENERAL, BE BRIEF. LONG SCRIBE NOTES, NO MATTER HOW WELL WRITTEN,
%**** ARE NEVER READ BY ANYBODY.


%%%%%%%%%%%%%%%%%%%%%%%%%%%%%%%%%%%%%%%%%%%%%%%%%%%%%%%%%%%%%%%%%%%%%%%%%%%%%%%%%%%%%%%%%%%%%%%%

%\emph{This solution is confidential and must not distributed outside the class without the explicit permission of the instructor.}


\begin{itemize}


\item[\bf 1.] In class, we learned about the dating problem and the 5-card trick. Prove that the trick protects the
privacy of Bob. 

\textbf{Solution:}
Without loss of generality, assume that Alice's decision is ``No date''; 
otherwise she can infer Bob's answer from the outcome. 
In other words, initially Alice will need to place her cards as $\heartsuit\clubsuit$. 
Thus the initial configuration is either $\heartsuit \clubsuit \heartsuit \heartsuit \clubsuit$
or $\heartsuit \clubsuit \heartsuit \clubsuit \heartsuit$. We view this as Bob's secret message (that Alice may have some \emph{a priori} knowledge), and our goal is to prove that the 
5-card trick does not give her any additional information. 


Recall that under the 5-card trick, Alice makes a private cut, which is a cyclic shift of the cards by $a \in \{0, 1, 2, 3, 4\}$ positions. 
Likewise, Bob's cut is a cyclic shift of the cards by $b \in \{0, 1, 2, 3, 4\}$ positions. 
Assume that Bob makes a uniformly random cut. %, that is, $b$ is uniformly distributed over $\{0, 1, 2, 3, 4\}$. 
From Alice's perspective, $a$ is known (and she might pick a number $a$ that she finds advantageous), 
but $b$ is secret and  uniformly distributed over $\{0, 1, 2, 3, 4\}$. 
Due to the two cuts by Alice and Bob, the cards are cyclic shifted by $(a + b) \bmod 5$ positions. 
In Alice's viewpoint, since $a, b \in \{0, 1, 2, 3, 4\}$ and $b$ is chosen uniformly at random and independent of $a$, 
the encryption key $c \gets (a + b) \bmod 5$ is also uniformly distributed over $\{0, 1, 2, 3, 4\}$. 
Let
\[ G = \{ \heartsuit \heartsuit \clubsuit \heartsuit \clubsuit, \; \clubsuit \heartsuit \heartsuit \clubsuit \heartsuit, \;
                    \heartsuit \clubsuit \heartsuit \heartsuit \clubsuit, \;
                    \clubsuit \heartsuit \clubsuit \heartsuit \heartsuit, \;
                    \heartsuit \clubsuit \heartsuit \clubsuit \heartsuit \} \enspace. 
	\]
	Note that $G$ contains both two possible initial configurations. 
For any configuration in $G$, if we cyclic shift it by $c \getsr \{0, 1, 2, 3, 4\}$ positions
then the resulting configuration is uniformly distributed over $G$. 
In other words, regardless of the initial configuration (that is, Bob's message),  from Alice's viewpoint, the final configuration (that is, Bob's ciphertext)
is uniformly distributed over $G$, and this distribution is independent of the message. 
Thus the ciphertext gives Alice no additional information about the~message. 









\end{itemize}

\end{document}





